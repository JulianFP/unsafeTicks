\section{How SafeTix Could Be Made Safer}
As demonstrated by Conduition and us, ticketing systems with dynamically changing offline tickets still present vulnerabilities, while also severely restricting and hindering the user's freedoms. Since these downsides of SafeTix and the frustrations they created were the reason why SafeTix got reverse engineered in the first place, we first want to discuss some alternatives to dynamically changing tickets that would tackle the same problems.

To prevent ticket theft, one alternative could be to educate users during and after the purchasing process 
about keeping their ticket barcode private. Additionally, the barcode could be hidden by default 
in the official app, with a button to reveal it only for a short period. To prevent scalping, 
the number of tickets per user account could be limited, or tickets could be released in smaller 
batches over an extended period.

If we were to ignore these strategies and focus solely on making SafeTix harder to crack, we might
consider using proper Digital Rights Management (DRM) systems to secure the tickets. However, 
this could be complex. A simpler approach would be to limit offline ticket use by rotating the 
ticket token \textit{server-side} every 24 hours. This would reduce convenience, as Ticketmaster 
would have to inform users that offline tickets must be downloaded no more than 24 hours before 
the event; otherwise, they would become invalid.

Ultimately, if tickets need to be stored on the client for long periods and work entirely offline, 
it is unlikely that extraction can be made completely impossible. However, making it more difficult 
than simply reading the ticket token from the Chrome web console would be a good first step.
