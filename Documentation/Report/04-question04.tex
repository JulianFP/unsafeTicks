\section{How SafeTix could be made safer}
First of all we would like to point out that we don't agree with the fundamental goal of SafeTix: Restricting the ticket ownership and selling opportunities to their own application and marketplace is incredibly anti-consumer and thus not a good approach to solve the problem of ticket theft and scalping.

Good alternative strategies would be to educate the user during and after the buying process that they should keep the ticket barcode secret, to hide the barcode by default in the official app and only have a button to reveal it for a short period of time. Alternative strategies against scalping could be to limit the amount of tickets that are allowed to be purchased by each user account or to make the tickets available for purchase in many small batches over a longer period of time.

However if we were to ignore this and just tried to make it more difficult to crack SafeTix we might look into proper Digital Rights Management (DRM) systems to secure the tickets. This however might prove quite complicated, easier strategies would be to limit the offline use of these tickets by for example rotating the ticket token \textit{server-side} every 24 hours or so. This would of course make the application less user-friendly since now Ticketmaster would have to tell their users to download the tickets at most 24 hours before the event if they want them to still work offline, not earlier since then the ticket would not be valid anymore once the user shows it at the venue.

At the end of the day if the ticket should be stored on the client for extended periods of times and work completely offline we don't think that there is a way to make it totally impossible for users to extract their tickets from the application. Making it more difficult than just reading the ticket token the chrome web console would be a good start.
