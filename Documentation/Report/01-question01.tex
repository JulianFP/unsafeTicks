\section{The original version of Safetix from Ticketmaster}
Ticketmaster is a leading American ticket sales and distribution company, well-known for selling tickets for a wide variety of events, 
including concerts, sports events, theater performances, and more. Their partners include major names such as Taylor Swift, 
the NFL, NHL, NBA, WWE, AEW, and popular theater productions like \textit{Harry Potter and the Cursed Child}, among many others. 

\subsection{Why they introduced Offline Tickets}
Ticketmaster introduced offline tickets to address one of the major pain points of digital ticketing: the dependency on an active internet connection. 
Attendees often face connectivity issues at crowded venues, where network overload or poor reception can make it difficult to access mobile tickets. 
By allowing tickets to be saved and accessed offline, Ticketmaster aimed to provide a more reliable and seamless experience for users, 
ensuring they can still access their tickets even in areas with spotty or no internet coverage. 
This move also helps to cater to customers who may have concerns about data privacy or simply prefer not to rely on continuous connectivity for event access.

\subsection{How they made Offline Tickets Secure}
To address the security concerns and challenges posed by offline tickets, Ticketmaster introduced SafeTix as an "improvement" over the traditional ticketing systems, a screenshot of Safetix can be seen in Figure
\ref{fig:app_barcode}.
\begin{figure}[H]
    \begin{center}        
    \includegraphics[scale = 0.3]{../figures/SafeTix_App_Barcode.png}
    \caption{A screenshot from Ticketmaster's app showing SafeTix in action \cite{ticketmaster_mobile_ticketing}}
    \label{fig:app_barcode}
    \end{center}
\end{figure}
While the old system allowed users to print tickets as PDFs or take screenshots, which could be easily shared or resold, SafeTix aims to eliminate these vulnerabilities by introducing a dynamic and rotating barcode. 
This rotating ticketing system was designed to provide a more secure, yet user-friendly alternative to static PDF tickets, and to combat issues like fraud and ticket scalping.

Ticketmaster markets SafeTix as an advanced solution to combat ticket fraud and scalping.
However, these marketing claims are often seen as exaggerated, as shown in Figure \ref{fig:safetix_quotes}.
\begin{figure}[H]
    \begin{center}
        \includegraphics[width=0.5\textwidth]{../figures/SafeTix_Quote_1.png}
    \end{center}
    \begin{center}
        \includegraphics[width=0.5\textwidth]{../figures/SafeTix_Quote_2.png}
    \end{center}
    \caption{Two quotes from Ticketmaster's and the NBA's FAQ pages about SafeTix \cite{ticketmaster_safetix_faq} \cite{nba_safetix_faq}}
    \label{fig:safetix_quotes}
\end{figure}
Some critics point out that the technology’s true purpose is more about limiting ticket transfers and keeping users within Ticketmaster's closed ecosystem.

\subsection{How it got reverse-engineered}
After inspecting the barcodes, Conduition found that the format was simple PDF417 barcodes that encode UTF-8 text. 
The blue bar sweeping across the barcode was just a CSS animation, 
not actually preventing screenshots, because PDF417 has error correction built in.

He noticed that the barcodes generated by the Ticketmaster web app contained four distinct pieces of data, 
delimited by colons—Base64-encoded data, followed by two six-digit numbers and a Unix timestamp. 
The first two six-digit numbers appeared similar to Time-based One-Time Passwords (TOTPs), 
likely generated from different secrets using the timestamp.

The static Base64 data looked like a random bearer token identifying the ticket holder and the ticket. 
When the barcode rotates every 15 seconds, only the two six-digit numbers and the timestamp change, but the Base64 data remains static. 
By debugging the Ticketmaster web app using Chrome DevTools, Conduition discovered an API request and its response, 
which included a token with a Base64-encoded JSON object containing the ticket's metadata. 
After decoding the token, Conduition identified the properties, including the bearer token and possible TOTP secrets.
The two TOTPs were constructed from an eventKey and customerKey, and the timestamp was used to verify the barcodes. 
He validated this by generating TOTPs using the identified secrets, matching the TOTPs in the barcode, 
and concluding that with the bearer token and the TOTP secrets, valid barcodes could be generated.
For the sake of not using Ticketsmasters App, he even made his own App that shows his current Barcode as can be seen in Figure \ref{fig:conduition_custom_app}.

\begin{figure}[H]
    \begin{center}
    \includegraphics[width=0.2\textwidth]{../figures/Conduition_custom_ticket_app.png}
    \caption{The Expo app TicketGimp Conduition wrote to store and render the extracted tickets \cite{reverse_engineering_ticketmaster}}
    \label{fig:conduition_custom_app}
    \end{center}
\end{figure}
